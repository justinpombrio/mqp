\documentclass[12pt]{article}
\usepackage{amsmath, amsthm, amssymb, verbatim}

\def\term{\emph}

\newcommand{\enc}[2]{\{|#1|\}_{#2}}
\newcommand{\pair}[2]{(#1, #2)}


\newtheorem{theorem}{Theorem}

\newcommand{\closure}{\odot}
\newcommand{\OR}{\bigvee}
\newcommand{\AND}{\bigwedge}
\newcommand{\true}{\top}
\newcommand{\false}{\bot}
\newcommand{\imp}{\Rightarrow}
\newcommand{\seq}[2]{#1 \ \vdash\  #2}
\newcommand{\aseq}[2]{#1 & \ \vdash\  #2}

\newcommand{\indOr}{\vee}

% Sorts
\newcommand{\Node}{\mbox{Node}}
\newcommand{\Msg}{\mbox{Term}}
\newcommand{\Strand}{\mbox{Strand}}

% Functions
\newcommand{\funcSig}[3]{#1 & : #2 \rightarrow #3}
\newcommand{\inv}[1]{#1^{-1}}
\newcommand{\invinv}[1]{#1^{-1 -1}}
\newcommand{\sk}[1]{\mbox{sk}(#1)}
\newcommand{\pk}[1]{\mbox{pk}(#1)}
\newcommand{\ltk}[1]{\mbox{lkt}(#1)}
\newcommand{\msg}[1]{\mbox{msg}(#1)}
\newcommand{\cat}[2]{#1 #2}
\newcommand{\strand}[1]{\mbox{strand}(#1)}

% Given Relations
\newcommand{\pred}[2]{#1 \leq #2}
\newcommand{\strPred}[2]{#1 \Rightarrow^* #2}
\newcommand{\outbound}[1]{+#1}
\newcommand{\inbound}[1]{-#1}
\newcommand{\non}[1]{\mbox{non } #1}
\newcommand{\uniq}[2]{#1 \mbox{ uniq-orig } #2}
\newcommand{\atomic}[1]{\mbox{atomic } #1}

% Defined Relations
\newcommand{\outPred}[2]{#1 \mbox{ out-pred } #2}
\newcommand{\subterm}[2]{#1 \sqsubseteq #2}
\newcommand{\visSubterm}[2]{#1 \sqsubseteq_v #2}
\newcommand{\avail}[1]{\mbox{avail } #1}
\newcommand{\unavail}[1]{\mbox{unavail } #1}
\newcommand{\comp}[1]{\mbox{compromised } #1}
\newcommand{\occ}[2]{#1 \mbox{ occ } #2}
\newcommand{\expl}[2]{#1 \mbox{ expl } #2}
\newcommand{\before}[2]{#1 \mbox{ sent-before } #2}



\begin{document}

Outline:
\begin{enumerate}
\item Formally define strand spaces / infiltrated skeletons
\item Write as a geometric theory
\item Partition the rules into soundness and completeness rules
\item Give inference rules (normalisation and efficiency and chaining
  -vs.- cuts)
\item Prove their equivalence to the completeness rules; show that
  each is redundant with respect to the other.
\item Find circumstances under which the chase fully terminates.
\end{enumerate}

Works so far:
\begin{enumerate}
\item Geometric Logic
\item The Chase
\item Normalization, Efficiency, Chaining
\end{enumerate}

Work to do:
\begin{enumerate}
\item CPSA algorithm
\item Translating protocol specs into geometric logic
\item Extensions
\end{enumerate}

\section{Motivation}

What are the advantages of geometric logic; why bother performing
cryptographic protocol analysis in this new domain?

Foremost, cryptographic protocol analysis is traditionally implemented
in imperitive programming languages. Switching to a declarative style
has the advantage of making correctness easier to verify, and the
system easier to reason about in general. First order logic is also a
natural way to express the definitions of the formalism used.

Second, a geometric theory is both flexible and extensible. Any
extension expressible in geometric logic can be added trivially; the
only implementation concern is the priority of the rules.

Third, the chase can perhaps shed some light on the working of
existing algorithms. It seems to mimic them in a way; there is perhaps
a close relation between the chase's search for minimal theories and
the algorithms' search for representative shapes.

\section{Strand Spaces}



%%%%%%%%%%%%%%%%%%%%%%%
\section{Geometric Logic}
%%%%%%%%%%%%%%%%%%%%%%%

A run of the chase \term{fully terminates} when it returns a finite
number of finite models and then halts.


\subsection{Definitions}
\begin{description}

\item[Signature] A set of constant symbols $c$, function symbols $f$
with arity, and relation symbols $R$ with arity.

\item[Model] (over a signature) Contains a non-empty \emph{universe}
from which a relation is assigned to each relation symbol, a function
to each function symbol, and an element to each constant symbol.

\item[Term] $a$ (over a signature) The closure of function symbols over
constants and variables.

\item[Homomorphism] $h$ (over a signature, from one model $A$ to another
$B$) A function from the universe of $A$ to that of $B$, obeying:

\begin{eqnarray*}
h(c_A) &=& c_B \\
h(f_A(a_1, ... a_n)) &=& f_B(h(a_1), ... h(a_n)) \\
R_A(a_1, ... a_n) &\imp& R_B(h(a_1), ... h(a_n))
\end{eqnarray*}

\item[Positive Existential Formula] (over a signature) The closure of
$\exists$, $\bigwedge$, and infinitary $\bigvee$ over terms, $\bot$,
and $\top$.

\item[Geometric Sequent] $\phi_1 \imp \phi_2$, where $\phi_1$ and
$\phi_2$ are positive existential formulas.  Written
$\seq{\phi_1}{\phi_2}$, with $\forall$ suppressed and ',' for 'and'.

\item[Geometric Theory] A finite conjunction of geometric sequents. Without
loss of generality, they can be written in the form
$$ \forall \vec{x}\; (\seq{D(\vec{x})}
{\OR_i(\exists \vec{y}_i\; .\; E_i(\vec{x}, \vec{y}_i))}) $$
where $D$ and $E_i$ are conjunctions of atomic formulas. [There must
be better terminology than ``strictly geom'' and ``weakly geom''.]

A relation is \emph{geometric}, or \emph{strictly geometric}, with
respect to a geometric theory if it can be expressed by adding
sequents to the theory.  It is \emph{weakly geometric} if it can be
expressed by expanding the signature as well as adding new sequents.

\item[Inductive Definition] An inductive definition for a relation is
a positive existential formula, lacking $\forall$, and with only
finite disjunctions, which may include the relation. It may always be
written in the form [check]:
$$ R(\vec{x}) \equiv \OR_i \exists
\vec{y}\; \AND_j \phi_{ij}(\vec{x}, \vec{y}) $$
where $\phi_ij$ is atomic and may contain '=' or $R$.

\end{description}


\subsection{Some Tricks in GL}
\begin{description}

\item[Negating a Sequent]
To negate a sequent,
$$ \seq{\phi(\vec{x})}
       {\OR_i \exists \vec{y}\; \psi_i(\vec{x}, \vec{y}, \vec{z})} $$
introduce constants $\vec{x_0}$ and $\vec{z_0}$. Force the LHS to be
true over $\vec{x_0}$,
$$ \seq{}{\phi(\vec{x_0})} $$
Then force each disjunct to be false over $\vec{x_0}$ and $\vec{z_0}$,
$$ \seq{\psi_i(\vec{x_0}, \vec{y}, \vec{z_0})}{} $$

\item[Replacing Functions with Relations and Equations]
Create a relation symbol $F$ for each function symbol $f$. Add a
sequent $\seq{F(\vec(x), a), F(\vec(x), b)}{a = b}$. Then replace each
sequent
$$ \seq{\phi(f(\vec{x}))}
       {\OR_i \exists \vec{y}\; \psi(f(\vec{x}, \vec{y}, \vec{z}))} $$
with
$$ \seq{F(\vec{x}, a), \phi(a)}
       {\OR_i \exists \vec{y}\; \exists b\;
              F(\vec{x}, \vec{y}, \vec{z}, b), \psi(b)} $$

\item[Replacing Constants with Functions and Equality]
For each constant symbol $c$, create a function symbol $C$, add the
sequent $\seq{C(x), C(y)}{x = y}$, and replace each occurrence of $c$
with $C$.

\item[Replacing Equality with a Relation]
Create a relation $E$. Make it symmetric and transitive,
\begin{align*}
\aseq{E(x, y)}{E(y, x)} \\
\aseq{E(x, y), E(y, z)}{E(x, z)}
\end{align*}
Then add sequents of the form
$$\seq{R(x), E(x, y)}{R(y)}$$
[Is this all that's neccessary?]

\item[Replacing Existential Quantifiers with Functions (Skolemization)]
$\exists x\; R(x, y)$ becomes $R(x, f(x)$

\end{description}


\subsection{Theorems}

\begin{theorem}
Inductive definitions are strictly geometric.
\end{theorem}

Expand the inductive definition as an infinite disjunction (first the
base cases, then the cases with one inductive step, etc.). Express
this as the right half of a geometric sequent.

\begin{theorem}
A geometric theory is satisfiable iff there is a run of the chase
which does not fail.
\end{theorem}

First, see that if a geometric theory is satisfiable then there is a
run of the chase which does not fail. Let $M$ be one of the models
satisfying the theory. Begin with an empty chase structure and a map
from its variables to variables in the model. As long as the chase has
not terminated, pick an unsatisfied sequent. One of the disjuncts of
the sequent, interpreted using the variable map, must be true in the
model. Expand the chase structure to make this disjunct true, adding
to the variable map as necessary. (But this doesn't prove there is a
\emph{computable} run of the chase which does not fail!)

Next, if a geometric theory is unsatisfiable then every run of the
chase fails. Clearly no run of the chase could succeed, because then
it would produce a model satisfying the theory. Nor could the chase
run forever, because then the theory would be satisfied by the model
formed by taking the union of the structures formed by the chase at
each iteration (more detail?).

\begin{theorem}
Let $T$ be geometric.  For any model $M$ of $T$ there is a run of the
chase that yields a model $N$ of $T$ such that there is a homomorphism
from $N$ to $M$.
\end{theorem}

See above.  The variable map is the homomorphism.

\begin{theorem}
[?] Suppose $T$ is geometric and can be expressed with no branching on
the right hand side of sequents.  If $T$ is satisfiable then no run of
the chase fails, and any result is a universal model.
\end{theorem}

As before, pick a model and use it as an oracle for the chase. Notice
that this time, there are no choices between disjuncts; only between
sequents. [...?]


\subsection{Parallelization}
It can be unsafe to coerce two rules in parallel. Consider the theory
\begin{verbatim}
    -> A & B & S
    A -> (P & S) | Q
    B -> P | (Q & S)
\end{verbatim}
The model ${A, B, S, P, Q}$ could be generated if instances of the
second and third rules were coerced in parallel, even though it is not
generated by any run of the chase!

Likewise, it can be dangerous to coerce two instances of a rule in
parallel. Consider the theory
\begin{verbatim}
    -> Exists x, y: A(x, y) & A(y, x)
    A(x, y) -> P(x) | P(y)
\end{verbatim}
Considering both instances of the second rule in parallel could
produce the model ${A(0, 1), P(0), P(1)}$, which again is not
generated by any run of the chase.

A sufficient condition for it to be safe to consider two rules in
parallel is when their right hand sides share no relations. In this
case, coercing the right hand side of 



\section{Implementation}

\subsection{Structure}
The geometric theory is partitioned into three sections:

\begin{enumerate}
\item Protocol Specification
\item Strand Space Axioms
\item Inference Rules
\end{enumerate}

\section{Target Theorems}

\begin{enumerate}
\item
Show that every model generated by the chase is, in fact, an
infiltrated skeleton.
\item
If $A$ is a realized infiltrated skeleton, then there exists a
homomorphism from some model returned by the chase to $A$.
\item
Show that the chase fully terminates under certain conditions.
\end{enumerate}

[Possible proof technique: assume the theory ends with ``completeness
  rules'' -- the rest of the axioms of the strand space formalisms --
  and then show them to be unnecessary.]

\section{Strand Spaces}

The \emph{strand space formalism} was developed as a method for
formally reasoning about cryptographic protocols. It distinguishes
between two different kinds of participants: regular participants and
an adversary. A single physical entity can be represented as multiple
regular strands if they play more than one role in the protocol.

Participants in a protocol run are represented by \emph{strands}, and
communicate with each other by sending and receiving messages. A
regular participant is represented by a regular strand and must follow
the protocol. The adversary is represented by zero or more adversary
strands, and can manipulate the messages that regular strands
send and receive.

The actions of both the regular participants and the advesary are
abstracted into message passing; this is assumed to be able to capture
all relevant information. For instance, if a private key is assumed
insecure and the advesary may be able to learn it, this could only be
represented by an advesary strand receiving the key. As such, every
strand consists of a (non-empty) sequence of message passing events
called \emph{nodes}. Each node either sends a message or receives a
message. A \emph{term} is any possible message.

In our representation of this formalism, nodes and terms are the
first-class objects. Strands are represented only indirectly through
nodes.

\begin{verbatim}
    Forall x. Term(x) | Node(x)
    Term(x) & Node(x) -> False
\end{verbatim}

\begin{verbatim}
    Send(n, t) -> Node(n) & Term(t)
    Recv(n, t) -> Node(n) & Term(t)

    Node(n) -> (Exists t. Send(n, t))
             | (Exists t. Recv(n, t))

    Send(n, s) & Send(n, t) -> s = t
    Recv(n, s) & Recv(n, t) -> s = t

    Send(n, s) & Recv(n, t) -> False
\end{verbatim}

\subsection{Messaging}

Terms are defined inductively over \emph{basic terms} as follows:

\begin{itemize}
\item Any basic term is a term.
\item The ciphertext $\enc{\tau_1}{\tau_2}$ is a term if the plaintext
  $\tau_1$ and the key $\tau_2$ are terms.
\item The pair $\pair{\tau_1}{\tau_2}$ is a term if $\tau_1$ and
  $\tau_2$ are terms.
\end{itemize}



\begin{comment}
	\subsection{Strand Spaces}

			A term $t$ is an \emph{ingredient} of another term $u$ if $u$ can
			be constructed from $t$ by repeatedly pairing with arbitrary terms
			and encrypting with arbitrary keys. A \emph{component} of a term
			$t$ is any term that can be retrieved simply by applying repeated
			unpairing operations to $t$ and is not a pair itself.

			A \emph{nonce} is a uniquely-originating basic term. A term $t$
			\emph{originates} on a node $n$ of a strand $s$ if $n$ is a
			sending node, $t$ is an ingredient of the message of $n$, and
			$t$ is not an ingredient of any previous node on $s$. A term is
			\emph{uniquely originating} if it originates on only one
			strand. A term is non-originating if it does not appear in any
			strand.

			If a regular participant generates a random fresh nonce, it
			will be assumed uniquely-originating because of the extreme
			unlikelihood of any other participant generating and
			originating the same value.

			Similarly, non-origination is helpful in describing private
			asymmetric keys that should never be sent as part of a message.

		\subsubsection{The Adversary}

			An adversary is included in the strand space
                        formalism to represent a worst-case situation,
                        where an attacker may control every point of
                        communication between regular participants.
                        The actions of the adversary are represented
                        by \emph{adversary strands}. Recall that
                        adversary strands are not bound by the rules
                        defined by the protocol; they manipulate
                        messages being sent and received by
                        non-adversarial strands.

			The capabilities of the adversary strands are given by the
			Dolev-Yao Threat Model \textbf{references}. The five possible
			operations that an adversary may perform are derived from both
			the Dolev-Yao Threat Model and the strand space formalism.
			These operations are:

			\begin{description}
			\item [pairing] the pairing of two terms
			\item [unpairing] the extraction of a term from a pair
			\item [encryption] given a key $k$ and a plaintext $m$, the construction of the ciphertext $\{|m|\}_k$ by encrypting $m$ with $k$
			\item [decryption] given a ciphertext $\{|m|\}_k$ and its decryption key $k^{-1}$, the extraction of the plaintext $m$
			\item [generation] the generation of an original term when that term is not assumed to be secure
			\end{description}

			Pairing and encryption are \emph{construction operations}.
			Decryption and unpairing are \emph{deconstruction operations}.

		\subsection{Paths}

			An \emph{edge} is a directional relationship between two nodes.
			The direct predecessor of a node within its strand is called its
			\emph{parent}. Not every node has a parent. Whenever a node $n$ has
			a parent $p$, there is an edge from $p$ to $n$. A \emph{link} is an
			edge from a sending node to a receiving node that have the same
			message. If there is a node $n$ with a link to a node $m$, $n$
			sends $m$ a message and $m$ receives it unaltered. The \emph{path}
			relation, written $x < y$, is the transitive closure of the edge
			relation. A node $n$ \emph{precedes} another node $m$ when there is
			a path from $n$ to $m$. In this formalism, events are partially
			ordered.

		\subsubsection{Normalisation / Simplification}

			Two simplifying assumptions can be made about the sequence of
			actions an adversary can perform without limiting their
			capabilities. These simplifications are called normalisation and
			efficiency. Guttman and Thayer \cite{GJF02} proved that adding
			these constraints does not limit the capabilities of an adversary.

			A protocol is \emph{efficient} if an adversary always takes a
			message from the earliest point at which it appears. To be precise,
			a protocol is efficient if, for every sending node $m$ and receiving
			adversary node $n$, if every component of $m$ is also a component of
			$n$, then there is no regular node $m'$ such that $m < m' < n$.

			A protocol is \emph{normal} if, for any path through adversary
			strands, the adversary always either performs a generation followed
			by zero or more construction operations or performs zero or more
			deconstruction operations followed by zero or more construction
			operations. This constraint limits redundancy.

			An important insight used in Cremers' algorithm \cite{Cremers08},
			which will be called \emph{chaining}, states that terms in messages
			received from an adversary strand always originate in a
			non-adversarial strand. In simpler terms, an adversary can not send
			a message to himself nor receive a message from himself.

	\subsection{The Problem}

		Some protocol researchers want to be able to programmatically reason
		about cryptographic protocols. A common technique for this is to find a
		set of essentially different classes of protocol runs, which are each a
		subset of all possible runs of the protocol. Together, these classes
		encompass every possible run of the protocol. This can be accomplished
		by finding minimal models of a geometric logic representation of the
		protocol. This happens to be precisely the problem the chase solves.

		\textbf{ things from Justin go here later }

	\subsection{The Solution: Minimal Models}

		Given a theory $T$, the jointly minimal models $\mathcal{M}$ that the
		chase outputs are representative of all models because there always
		exists a homomorphism from some model $\mathbb{M} \in \mathcal{M}$ to
		any model that satisfies $T$. Every model $\mathbb{N}$ that satisfies
		$T$ describes a possible run of the protocol. It also indirectly
		represents a larger class of runs, specifically the set of all runs of
		the protocol that are described by a model $\mathbb{P}$ such that
		$\mathbb{P} \models T$ and $\mathbb{N} \preceq \mathbb{P}$.

		Since the set of all models output of the chase is jointly minimal,
		they represents every possible run of the protocol. Finding every
		possible run of the protocol would be prohibitive because there are
		infinitely many. Fortunately, the chase may halt with a finite result,
		providing a model that is representative of a class of protocol runs.
		Fortunately, the deterministic chase implementation may output a finite
		number of models before halting. These models are jointly minimal, and
		therefore represent all runs of the protocol.

	\subsection{Designing An Analogous Theory}

		In order to create a geometric theory describing a protocol, the
		formul{\ae} that define strand spaces, normilisation, efficiency, and
		chaining must be derived. The formul{\ae} defining the protocol must be
		combined with this scaffolding to create a theory that can be used to
		infer the possible runs of the protocol.

		The \emph{half-duplex protocol} was chosen to be used as an example.
		This protocol involves two participants, Alice and Bob, and specifies
		that the following actions take place:

		\begin{enumerate}
		\item Alice sends Bob a nonce that she generated, encrypted with Bob's public key
		\item Bob receives the encrypted nonce
		\item Bob replies to Alice with the decrypted nonce
		\item Alice receives Bob's message
		\end{enumerate}

		A visual representation of the protocol, as modeled in the strand space
		formalism, can be seen in Figure \ref{fig:half-duplex}, Appendix
		\ref{sec:appendixHalfDuplex}.

		The geometric logic rules that model this protocol were generated
		manually by direct translation into geometric logic. Ideally, the
		process of generating geometric logic formul{\ae} from protocols should
		be done automatically.

        %% The chase can be used for protocol analysis. A common technique for the analysis
	%% of protocols involves identifying the \emph{essentially different} runs of
	%% the protocol. These essentially different protocol runs are analogous to
	%% minimal models. When a protocol is described using geometric logic, the
	%% chase can find such minimal models. The protocol can then be analysed for
	%% characteristics such as the existence of security violations or other
	%% unexpected behaviour.


\end{comment}
\end{document}
