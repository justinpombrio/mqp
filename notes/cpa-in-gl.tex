\documentclass[12pt]{article}
\usepackage{amsthm}

\def\term{\emph}

\newcommand{\compose}[2]{#1 \circ #2}
\newcommand{\run}[2]{\term{run}(#1,#2)}


\begin{document}

Works so far:
\begin{enumerate}
\item Geometric Logic
\item The Chase
\item Normalization, Efficiency, Chaining
\end{enumerate}

Work to do:
\begin{enumerate}
\item CPSA algorithm
\item Translating protocol specs into geometric logic
\item Extensions
\end{enumerate}

\section{Motivation}

What are the advantages of geometric logic; why bother performing
cryptographic protocol analysis in this new domain?

Foremost, cryptographic protocol analysis is traditionally implemented
in imperitive programming languages. Switching to a declarative style
has the advantage of making correctness easier to verify, and the
system easier to reason about in general. First order logic is also a
natural way to express the definitions of the formalism used.

Second, a geometric theory is both flexible and extensible. Any
extension expressible in geometric logic can be added trivially; the
only implementation concern is the priority of the rules.

Third, the chase can perhaps shed some light on the working of
existing algorithms. It seems to mimic them in a way; there is perhaps
a close relation between the chase's search for minimal theories and
the algorithms' search for representative shapes.

\section{Geometric Logic}

A run of the chase \term{fully terminates} when it returns a finite
number of finite models and then halts.

\section{Implementation}

\subsection{Structure}
The geometric theory is partitioned into three sections:

\begin{enumerate}
\item Protocol Specification
\item Strand Space Axioms
\item Inference Rules
\end{enumerate}

\section{Target Theorems}

\begin{enumerate}
\item
Show that every model generated by the chase is, in fact, an
infiltrated skeleton.
\item
If $A$ is a realized infiltrated skeleton, then there exists a
homomorphism from some model returned by the chase to $A$.
\item
Show that the chase fully terminates under certain conditions.
\end{enumerate}

[Possible proof technique: assume the theory ends with ``completeness
  rules'' -- the rest of the axioms of the strand space formalisms --
  and then show them to be unnecessary.]

\end{document}
