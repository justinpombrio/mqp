\documentclass{article}

\def \compose {\circ}


\begin{document}

\title{Graph Homomorphisms}
\author{Justin Pombrio}
\maketitle

\section{Graphs}

A \emph{graph} is a set of \emph{vertices} along with a set of
unordered pairs of distinct vertices called \emph{edges}. A
\emph{digraph} is like a graph, but it's edges are ordered pairs. A
\emph{relational structure}, or just \emph{structure}, has a set of
\emph{vertices}, a set of \emph{relations} with natural \emph{arity},
and a set of $n$-tuples of vertices for each relation of arity $n$.

Unless otherwise mentioned, the following definitions and theorems
should apply equally well to all three kinds of objects: graphs,
digraphs, and structures.

\section{Cores}

\begin{description}
\item[Homomorphism] A \emph{homomorphism from $G$ to $H$} is a
  function $\phi$ from the vertices of $G$ to the vertices of $H$ that
  preserves edges. That is, if $e$ is an edge of $G$, then the edge
  formed by applying $\phi$ to each component of $e$ is an edge of
  $H$.
\item[Retract] A \emph{retract}, or \emph{folding}, of $G$ is an
  endomorphism $\phi$ onto a subgraph $H$ of $G$ such that $x \in H$
  implies $\phi(x) = x$.
\item[Core] An object for which every endomorphism is also an
  automorphism.
\item[Antichain] A set of objects unrelated by homomorphisms.
\end{description}

\subsection{Results from [1]}
\begin{enumerate}
\item A homomorphism equivalence class has at most one core.
\item A core is uniquely represented as an antichain of connected
  cores.
\item A graph $G$ is uniquely represented as the infinite sequence
  $|Hom(F_i, G)|$ for any enumeration of all finite graphs $F_i$.
\end{enumerate}

\subsection{Proofs}
\begin{enumerate}
\item If an equivalence class has two cores, then there are
  homomorphisms from each to the other, $\phi$ and $\phi'$. Consider
  the compositions $\phi \compose \phi'$ and $\phi' \compose
  \phi$. The first is an endomorphism from the first object to itself,
  and hence an automorphism, and the second is an endomorphism from
  second object to itself, hence an automorphism. Since both $\phi
  \compose \phi'$ and $\phi' \compose \phi$ are bijections, so are
  $\phi$ and $\phi'$. Now we can show that $\phi$ is an
  isomorphism. We already know that it is a bijective homomorphism, so
  we need only show that it's inverse $\phi^{-1}$ is a homomorphism.
  $\phi^{-1}$ is equal to $(\phi' \compose \phi)^{-1} \compose \phi'$,
  which is the composition of an automorphism and a homomorphism,
  which is a homomorphism. Thus $\phi$ is an isomorphism and the
  equivalence class's cores are isomorphic.
\item Every core is the disjoint union of some connected
  components. Each component must be a core, or else it would have an
  endomorphism which is not an automorphism and so would the whole
  object. Likewise, there can be no homomorphism between components,
  since it could be used to construct an endomorphism which is not an
  automorphism by mapping one component to the other, and every other
  component to itself. Thus the components of any core (which are
  themselves connected cores) form an antichain.
\end{enumerate}

In [2], Bauslaugh points out that cores ought be defined as graphs for
which every endomorphism is an automorphism, and \emph{not} as a
vertex-minimal member of a graph homomorphism equivalence class as
suggested in [1]. For finite graphs, these definitions are equivalent,
but for infinite graphs only the latter results in cores being
unique. Consider, for instance, the (countably) infinite graph with
vertices ${0, 1, 2, ...}$ and edges ${(x, y) | x < y}$. Under the
vertex-minimal core definition, this graph has an infinite number of
cores, given by the subgraphs induced by ${n, n+1, n+2, ...}$ for any
$n \geq 1$. These are in the same homomorphism equivalence class -- a
forward homomorphism maps $x$ to $x + n$, and a reverse homomorphism
maps $x$ to $x$. And each core is indeed vertex minimal -- they each
have infinitely many vertices, and there is no homomorphism to any
finite graph, since that graph would have to include a clique of every
order.

\emph{A jointly universal set of relational structures may have more
  than one core.} Take as an example the set consisting of a triangle
and a Grotzsch Graph.

\begin{enumerate}
\item Determining whether $G$ is $H$-colorable is NP complete for
  fixed $H$ and varying $G$.
\item If there is a homomorphism from $G$ to $H$, what can you say
  about the existence of cores of $G$ and $H$?
\item Is the image of every endomorphism isomorphic to a retract?
\end{enumerate}

[1]: Peter J. Cameron. Graph homomorphisms (class notes). September 2006. http://www.maths.qmul.ac.uk/~pjc/csgnotes/hom1.pdf

[2]: Bruce Lloyd Bauslaugh. Homomorphisms of infinite directed
graphs. December 1994. Simon Fraser University.

\end{document}


\begin{comment}
A \emph{digraph} consists of a set of \emph{vertices} along with a set
of (ordered) pairs of vertices called \emph{edges}. There are two
equivalent ways to define graphs. They may be defined similarly to
digraphs, except with edges being \emph{unordered} pairs of
vertices. Or they may be viewed as a special case of digraph in which
the edge relation is symmetric and irreflexive. We will use the first
definition [this could change].
\end{comment}
