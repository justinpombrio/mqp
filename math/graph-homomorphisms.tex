\documentclass{article}

\begin{document}

\title{Graph Homomorphisms}
\author{Justin Pombrio}
\maketitle

\section{Cores}

A \emph{graph homomorphism from $G$ to $H$} is a function $h$ from the
vertices of $G$ to the vertices of $H$ that preserves edges.  That is,
if $(x, y) \in E(G)$, then $(h(x), h(y)) \in E(H)$.

\begin{description}
\item[Retract (folding)] An endomorphism $h$ onto a subgraph $H$ such that $x
  \in H$ implies $h(x) = x$.
\item[Core] The (unique) vertex-minimal graph in a homomorphism
  equivalence class.
\item[Antichain] A set of objects unrelated by homomorphisms.
\end{description}

\subsection{Results from [1]}
\begin{enumerate}
\item A graph homomorphism is uniquely represented as the composition
  of a retract relation and a subgraph relation.
\item A homomorphism equivalence class is uniquely represented as a
  core.
\item A core is uniquely represented as an antichain of connected
  cores.
\item A graph $G$ is uniquely represented as the infinite sequence
  $|Hom(F_i, G)|$ for any enumeration of all finite graphs $F_i$.
\end{enumerate}

\subsection{Proofs}
\begin{enumerate}
\item Given a homomorphism $h$ from $G$ to $H$, its retract is the
  image of $G$ under $H$, which is a subgraph of $H$. Given a retract
  of $G$ which is a subgraph of $H$, their composition will be a
  homomorphism because both preserve edges.
\item Every homomorphism equivalence class has a core. If an
  equivalence class has two cores, then there are homomorphism from
  each to the other, which define a bijection, which must preserve
  edges both ways. Thus the cores are isomorphic.
\item Every core is the disjoint union of some connected
  components. Each component must be a core, or else there would be a
  homomorphism from the whole graph to a smaller graph. Likewise,
  there can be no homomorphism between components, or else there would
  be an endomorphism from the whole graph to a proper subgraph of
  it. Thus the components of any core form an antichain.
\end{enumerate}

[1]: Peter J. Cameron. Graph homomorphisms (class notes). September 2006. http://www.maths.qmul.ac.uk/~pjc/csgnotes/hom1.pdf\&ei=mGLGTKbnBIS8lQfn08nhAQ\&usg=AFQjCNGzeaz6lktEnqV0CabYPW-IhTDbkw

\end{document}
