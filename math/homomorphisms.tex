\documentclass{article}

\newcommand{\union}{\cup}
\newcommand{\term}{\emph}
\newcommand{\repr}[1]{\chi(#1)}
\newcommand{\homo}[2]{#1 \rightarrow #2}
\newcommand{\compose}{\o}


\begin{document}

\title{Relational Structure Homomorphisms}
\author{Justin Pombrio}
\maketitle


\section{Digraphs}
  A directed graph, or \term{digraph}, consists of a set of objects
  called \term{vertices} along with a set of ordered pairs of vertices
  called \term{arcs}.

  A \term{homomorphism} between two digraphs is a function $\phi$ from
  the vertices of one to the vertices of the other which preserves
  arcs.  More formally, a homomorphism $\phi$ from $G_1$ to $G_2$ is a
  function from the vertices of $G_1$ to the vertices of $G_2$ such
  that if $(x, y)$ is an arc in $G_1$, then $(\phi(x), \phi(y))$ must
  be an arc in $G_2$.  For convenience, let us overload homomorphisms
  $\phi$ by writing $\phi((x, y))$ to mean $(\phi(x), \phi(y))$.

\section{Structures}
  A \term{signature} is a pair $(S, A)$, where $S$ is a set of objects
  called \term{relation symbols}, and $A$ is a function from relation
  symbols to their \term{arity}, a natural number.

  A relational structure, or just \term{structure}, over a given
  signature $(S, A)$ is a set of objects called \term{vertices} and a set
  of \term{facts}.  Each fact is a relation symbol $R$ from $S$ applied
  to an $n$-tuple of vertices, where $n$ is $A(R)$.  A fact with
  relation symbol $R$ applied to vertices $x_1, ..., x_n$ is written
  $R(x_1, ..., x_n)$.

\section{Homomorphisms}
  The homomorphism relation is \emph{reflexive} because the identity
  function is always a homomorphism from a thing to itself.  It is
  also \emph{transitive}, because the composition of a homomorphism
  from $X$ to $Y$ and a homomorphism from $Y$ to $Z$ is itself a
  homomorphism from $X$ to $Z$.  Thus the homomorphism relation forms
  a \emph{quasi-order}.  But homomorphisms are not, in general,
  antisymmetric (that is, they do not neccessarily form a
  \emph{partial order}).  Take, for instance, the digraphs $(\{x, y,
  z\}, \{(x, y), (x, z)\})$ and $(\{a, b\}, \{(a, b)\})$ (a path of
  length 2 and a path of length 1).  These digraphs are not
  isomorphic, yet there are homomorphisms from each to the other.
  When there are homomorphisms from each of two things to the other,
  we call them \term{homomorphically equivalent}.

  Write $h(R(x_1, ..., x_n))$ for $R(h(x_1), ..., h(x_n))$.

\section{Representing Structures as Colored Digraphs}
  We will define a function $\repr$ which takes a structure with
  relations symbols $S$, whose maximum arity is $k$, and produces a
  colored digraph over the colors $S \union \{1, ... k\}$.  The
  vertices of $\repr{M}$ are the vertices and edges of $M$.  And for
  every fact $f = R(x_1, ..., x_n)$ in $M$, $\repr{M}$ has an edge
  $(f, f)$ of color $R$ and edges $(f, x_i)$ of color $i$ for each
  $i$.

  \emph{Claim:}
  \[\homo{M_1}{M_2} \mbox{ iff } \homo{\repr{M_1}}{\repr{M_2}} \]

  \emph{Proof:} We will show that both the forward and reverse
  implications hold.  For the foward implication, assume that $h$ is a
  homomorphism from $M_1$ to $M_2$.  We will construct the
  homomorphism $h'$ from $\repr{M_1}$ to $\repr{M_2}$ as follows:
  $h'(x)$ = $h(x)$ for all vertices $x$ of $M_1$, and $h'(R(x_1, ...,
  x_n))$ = $h(R(x_1, ..., x_n))$ = $R(h(x_1), ..., h(x_n))$ for each
  fact $R(x_1, ..., x_n)$ of $M_1$.  Now we must show that $h'$
  preserves edges.  First consider the edges of the form $(f, f)$ and
  color $R$, where $f = R(x_1, ..., x_n)$ is a fact of $M_1$.  Each
  such edge will be mapped to $(h(f), h(f))$ of color $R$.  This is
  indeed an edge of $\repr{M_2}$, because $h(f)$ must be a fact in
  $M_2$.  Now consider the other edges, which have form $(f, x_i)$ of
  color $i$, where $f = R(x_1, ..., x_n)$ and $i = 1..n$.  Each will
  be mapped to $(h(f), h(x_i))$ of color $i$.  This is an edge of
  $\repr{M_2}$ because $h(f) = R(h(x_1), ..., h(x_i), ..., h(x_n))$ is
  a fact of $M_2$.

  For the reverse implication, assume that $h$ is a homomorphism from
  $\repr{M_1}$ to $\repr{M_2}$.  The homomorphism $h'$ from $M_1$ to
  $M_2$ is just the restriction of $h$ onto the vertices of $M_1$.  We
  must show that $h'$ preserves facts.  Any fact $f = R(x_1, ...,
  x_n)$ of $M_1$ will be mapped to $R(h(x_1), ..., h(x_n))$.  Since
  $(f, f)$ of color $R$ is an edge of $\repr{M_1}$, $(h(f), h(f))$ of
  color $R$ must be an edge of $\repr{M_2}$.  Likewise, there must be
  edges $(f, x_i)$ of color $i$ for each $i$.  The only way these
  $i+1$ edges could be present in $\repr{M_2}$ is if $R(h(x_1), ...,
  h(x_n))$ was a fact in $M_2$.

[Older notes from here on out; some falsehoods present]
\section{Older Notes}

\begin{description}
\item[Homomorphism] A \emph{homomorphism from $G$ to $H$} is a
  function $\phi$ from the vertices of $G$ to the vertices of $H$ that
  preserves edges. That is, if $e$ is an edge of $G$, then the edge
  formed by applying $\phi$ to each component of $e$ is an edge of
  $H$.
\item[Retract] A \emph{retract}, or \emph{folding}, of $G$ is an
  endomorphism $\phi$ onto a subgraph $H$ of $G$ such that $x \in H$
  implies $\phi(x) = x$.
\item[Core] An object for which every endomorphism is also an
  automorphism.
\item[Antichain] A set of objects unrelated by homomorphisms.
\end{description}

\subsection{Results from [1]}
\begin{enumerate}
\item A homomorphism equivalence class has at most one core.
\item A core is uniquely represented as an antichain of connected
  cores.
\item A graph $G$ is uniquely represented as the infinite sequence
  $|Hom(F_i, G)|$ for any enumeration of all finite graphs $F_i$.
\end{enumerate}

\subsection{Proofs}
\begin{enumerate}
\item If an equivalence class has two cores, then there are
  homomorphisms from each to the other, $\phi$ and $\phi'$. Consider
  the compositions $\phi \compose \phi'$ and $\phi' \compose
  \phi$. The first is an endomorphism from the first object to itself,
  and hence an automorphism, and the second is an endomorphism from
  second object to itself, hence an automorphism. Since both $\phi
  \compose \phi'$ and $\phi' \compose \phi$ are bijections, so are
  $\phi$ and $\phi'$. Now we can show that $\phi$ is an
  isomorphism. We already know that it is a bijective homomorphism, so
  we need only show that it's inverse $\phi^{-1}$ is a homomorphism.
  $\phi^{-1}$ is equal to $(\phi' \compose \phi)^{-1} \compose \phi'$,
  which is the composition of an automorphism and a homomorphism,
  which is a homomorphism. Thus $\phi$ is an isomorphism and the
  equivalence class's cores are isomorphic.
\item Every core is the disjoint union of some connected
  components. Each component must be a core, or else it would have an
  endomorphism which is not an automorphism and so would the whole
  object. Likewise, there can be no homomorphism between components,
  since it could be used to construct an endomorphism which is not an
  automorphism by mapping one component to the other, and every other
  component to itself. Thus the components of any core (which are
  themselves connected cores) form an antichain.
\end{enumerate}

In [2], Bauslaugh points out that cores ought be defined as graphs for
which every endomorphism is an automorphism, and \emph{not} as a
vertex-minimal member of a graph homomorphism equivalence class as
suggested in [1]. For finite graphs, these definitions are equivalent,
but for infinite graphs only the latter results in cores being
unique. Consider, for instance, the (countably) infinite graph with
vertices $\{0, 1, 2, ...\}$ and edges $\{(x, y) | x < y\}$. Under the
vertex-minimal core definition, this graph has an infinite number of
cores, given by the subgraphs induced by $\{n, n+1, n+2, ...\}$ for
any $n \geq 1$. These are in the same homomorphism equivalence class
-- a forward homomorphism maps $x$ to $x + n$, and a reverse
homomorphism maps $x$ to $x$. And each core is indeed vertex minimal
-- they each have infinitely many vertices, and there is no
homomorphism to any finite graph, since that graph would have to
include a clique of every order.

\emph{A jointly universal set of relational structures may have more
  than one core.} Take as an example the set consisting of a triangle
and a Grotzsch Graph.

\begin{enumerate}
\item Determining whether $G$ is $H$-colorable is NP complete for
  fixed $H$ and varying $G$.
\item If there is a homomorphism from $G$ to $H$, what can you say
  about the existence of cores of $G$ and $H$?
\item Is the image of every endomorphism isomorphic to a retract?
\end{enumerate}

[1]: Peter J. Cameron. Graph homomorphisms (class notes). September 2006. http://www.maths.qmul.ac.uk/~pjc/csgnotes/hom1.pdf

[2]: Bruce Lloyd Bauslaugh. Homomorphisms of infinite directed
graphs. December 1994. Simon Fraser University.

\end{document}

