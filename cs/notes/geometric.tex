\documentclass{article}

\usepackage{amsmath, amsthm, amssymb}

\begin{document}
\title{Geometric Logic & Strand Spaces}
\author{Justin Pombrio}
\date{5/28/10}

\newtheorem{theorem}{Theorem}

\newcommand{\closure}{\odot}
\newcommand{\OR}{\bigvee}
\newcommand{\AND}{\bigwedge}
\newcommand{\true}{\top}
\newcommand{\false}{\bot}
\newcommand{\imp}{\Rightarrow}
\newcommand{\seq}[2]{#1 \ \vdash\  #2}
\newcommand{\aseq}[2]{#1 & \ \vdash\  #2}

\newcommand{\indOr}{\vee}

% Sorts
\newcommand{\Node}{\mbox{Node}}
\newcommand{\Msg}{\mbox{Term}}
\newcommand{\Strand}{\mbox{Strand}}

% Functions
\newcommand{\funcSig}[3]{#1 & : #2 \rightarrow #3}
\newcommand{\inv}[1]{#1^{-1}}
\newcommand{\invinv}[1]{#1^{-1 -1}}
\newcommand{\sk}[1]{\mbox{sk}(#1)}
\newcommand{\pk}[1]{\mbox{pk}(#1)}
\newcommand{\ltk}[1]{\mbox{lkt}(#1)}
\newcommand{\msg}[1]{\mbox{msg}(#1)}
\newcommand{\cat}[2]{#1 #2}
\newcommand{\enc}[2]{\{#1\}_{#2}}
\newcommand{\strand}[1]{\mbox{strand}(#1)}

% Given Relations
\newcommand{\pred}[2]{#1 \leq #2}
\newcommand{\strPred}[2]{#1 \Rightarrow^* #2}
\newcommand{\outbound}[1]{+#1}
\newcommand{\inbound}[1]{-#1}
\newcommand{\non}[1]{\mbox{non } #1}
\newcommand{\uniq}[2]{#1 \mbox{ uniq-orig } #2}
\newcommand{\atomic}[1]{\mbox{atomic } #1}

% Defined Relations
\newcommand{\outPred}[2]{#1 \mbox{ out-pred } #2}
\newcommand{\subterm}[2]{#1 \sqsubseteq #2}
\newcommand{\visSubterm}[2]{#1 \sqsubseteq_v #2}
\newcommand{\avail}[1]{\mbox{avail } #1}
\newcommand{\unavail}[1]{\mbox{unavail } #1}
\newcommand{\comp}[1]{\mbox{compromised } #1}
\newcommand{\occ}[2]{#1 \mbox{ occ } #2}
\newcommand{\expl}[2]{#1 \mbox{ expl } #2}
\newcommand{\before}[2]{#1 \mbox{ sent-before } #2}


%%%%%%%%%%%%%%%%%%%%%%%
\section{Notes}
%%%%%%%%%%%%%%%%%%%%%%%

\subsection{Definitions}
\begin{description}

\item[Signature] A set of constant symbols $c$, function symbols $f$
with arity, and relation symbols $R$ with arity.

\item[Model] (over a signature) Contains a non-empty \emph{universe}
from which a relation is assigned to each relation symbol, a function
to each function symbol, and an element to each constant symbol.

\item[Term] $a$ (over a signature) The closure of function symbols over
constants and variables.

\item[Homomorphism] $h$ (over a signature, from one model $A$ to another
$B$) A function from the universe of $A$ to that of $B$, obeying:

\begin{eqnarray*}
h(c_A) &=& c_B \\
h(f_A(a_1, ... a_n)) &=& f_B(h(a_1), ... h(a_n)) \\
R_A(a_1, ... a_n) &\imp& R_B(h(a_1), ... h(a_n))
\end{eqnarray*}

\item[Positive Existential Formula] (over a signature) The closure of
$\exists$, $\bigwedge$, and infinitary $\bigvee$ over terms, $\bot$,
and $\top$.

\item[Geometric Sequent] $\phi_1 \imp \phi_2$, where $\phi_1$ and
$\phi_2$ are positive existential formulas.  Written
$\seq{\phi_1}{\phi_2}$, with $\forall$ suppressed and ',' for 'and'.

\item[Geometric Theory] A finite conjunction of geometric sequents. Without
loss of generality, they can be written in the form
$$ \forall \vec{x}\; (\seq{D(\vec{x})}
{\OR_i(\exists \vec{y}_i\; .\; E_i(\vec{x}, \vec{y}_i))}) $$
where $D$ and $E_i$ are conjunctions of atomic formulas. [There must
be better terminology than ``strictly geom'' and ``weakly geom''.]

A relation is \emph{geometric}, or \emph{strictly geometric}, with
respect to a geometric theory if it can be expressed by adding
sequents to the theory.  It is \emph{weakly geometric} if it can be
expressed by expanding the signature as well as adding new sequents.

\item[Inductive Definition] An inductive definition for a relation is
a positive existential formula, lacking $\forall$, and with only
finite disjunctions, which may include the relation. It may always be
written in the form [check]:
$$ R(\vec{x}) \equiv \OR_i \exists
\vec{y}\; \AND_j \phi_{ij}(\vec{x}, \vec{y}) $$
where $\phi_ij$ is atomic and may contain '=' or $R$.

\end{description}


\subsection{Some Tricks in GL}
\begin{description}

\item[Negating a Sequent]
To negate a sequent,
$$ \seq{\phi(\vec{x})}
       {\OR_i \exists \vec{y}\; \psi_i(\vec{x}, \vec{y}, \vec{z})} $$
introduce constants $\vec{x_0}$ and $\vec{z_0}$. Force the LHS to be
true over $\vec{x_0}$,
$$ \seq{}{\phi(\vec{x_0})} $$
Then force each disjunct to be false over $\vec{x_0}$ and $\vec{z_0}$,
$$ \seq{\psi_i(\vec{x_0}, \vec{y}, \vec{z_0})}{} $$

\item[Replacing Functions with Relations and Equations]
Create a relation symbol $F$ for each function symbol $f$. Add a
sequent $\seq{F(\vec(x), a), F(\vec(x), b)}{a = b}$. Then replace each
sequent
$$ \seq{\phi(f(\vec{x}))}
       {\OR_i \exists \vec{y}\; \psi(f(\vec{x}, \vec{y}, \vec{z}))} $$
with
$$ \seq{F(\vec{x}, a), \phi(a)}
       {\OR_i \exists \vec{y}\; \exists b\;
              F(\vec{x}, \vec{y}, \vec{z}, b), \psi(b)} $$

\item[Replacing Constants with Functions and Equality]
For each constant symbol $c$, create a function symbol $C$, add the
sequent $\seq{C(x), C(y)}{x = y}$, and replace each occurrence of $c$
with $C$.

\item[Replacing Equality with a Relation]
Create a relation $E$. Make it symmetric and transitive,
\begin{align*}
\aseq{E(x, y)}{E(y, x)} \\
\aseq{E(x, y), E(y, z)}{E(x, z)}
\end{align*}
Then add sequents of the form
$$\seq{R(x), E(x, y)}{R(y)}$$
[Is this all that's neccessary?]

\item[Replacing Existential Quantifiers with Functions (Skolemization)]
$\exists x\; R(x, y)$ becomes $R(x, f(x)$

\end{description}


\subsection{Theorems}

\begin{theorem}
Inductive definitions are strictly geometric.
\end{theorem}

Expand the inductive definition as an infinite disjunction (first the
base cases, then the cases with one inductive step, etc.). Express
this as the right half of a geometric sequent.

\begin{theorem}
A geometric theory is satisfiable iff there is a run of the chase
which does not fail.
\end{theorem}

First, see that if a geometric theory is satisfiable then there is a
run of the chase which does not fail. Let $M$ be one of the models
satisfying the theory. Begin with an empty chase structure and a map
from its variables to variables in the model. As long as the chase has
not terminated, pick an unsatisfied sequent. One of the disjuncts of
the sequent, interpreted using the variable map, must be true in the
model. Expand the chase structure to make this disjunct true, adding
to the variable map as necessary. (But this doesn't prove there is a
\emph{computable} run of the chase which does not fail!)

Next, if a geometric theory is unsatisfiable then every run of the
chase fails. Clearly no run of the chase could succeed, because then
it would produce a model satisfying the theory. Nor could the chase
run forever, because then the theory would be satisfied by the model
formed by taking the union of the structures formed by the chase at
each iteration (more detail?).

\begin{theorem}
Let $T$ be geometric.  For any model $M$ of $T$ there is a run of the
chase that yields a model $N$ of $T$ such that there is a homomorphism
from $N$ to $M$.
\end{theorem}

See above.  The variable map is the homomorphism.

\begin{theorem}
[?] Suppose $T$ is geometric and can be expressed with no branching on
the right hand side of sequents.  If $T$ is satisfiable then no run of
the chase fails, and any result is a universal model.
\end{theorem}

As before, pick a model and use it as an oracle for the chase. Notice
that this time, there are no choices between disjuncts; only between
sequents. [...?]


%%%%%%%%%%%%%%%%%%%%%%%
\section{Skeletons in Geometric Logic}
%%%%%%%%%%%%%%%%%%%%%%%
\subsection{Signature}

\begin{itemize}

\item Sorts: \\
$\Strand$ $A$, $B$, $C$, $D$ \\
$\Node$ $n$, $m$, $p$ \\
$\Msg$ $s$, $t$, $g$, $h$, $k$

\item Functions: \\
$\msg{}$ : $\Node \to \Msg$ \\
$\inv{k}$ : $\Msg \to \Msg$ \\
$\strand{}$ : $\Node \to \Strand$ \\
$\cat{g}{h}$ : $\Msg, \Msg \to \Msg$ \\
$\enc{g}{k}$ : $\Msg, \Msg \to \Msg$ \\
$\sk{}$ : $\Strand \to \Msg$ \\
$\pk{}$ : $\Strand \to \Msg$ \\
$\ltk{}{}$ : $\Strand, \Strand \to \Msg$

\item Given Relations: \\
$\strPred{\Node}{\Node}$ \\
$\pred{\Node}{\Node}$ \\
$\outbound{\Node}$ \\
$\inbound{\Node}$ \\
$\non{\Msg}$ \\
$\uniq{\Msg}{\Node}$ \\
($\atomic{\Msg}$)

\item Defined Relations: \\
$\outPred{\Node}{\Node}$ \\
$\subterm{\Msg}{\Msg}$ (within) \\
$\visSubterm{\Msg}{\Msg}$ (visibly within) \\
$\avail{\Msg}$ \\
($\occ{\Msg}{\Node}$) \\
$\before{\Msg}{\Node}$ \\
$\expl{\Msg}{\Node}$

\end{itemize}

The predecessor relation forms a total order, and strand predecessors
form a partial order.

(The rules for keys were taken from \emph{Shapes: Surveying Crypto Protocol Runs}, section 2.1.)



%%%%%%%%%%%%%%%%%%%%%%
\subsection{Axioms}
%%%%%%%%%%%%%%%%%%%%%%

\begin{description}

\item[Atomic]

\begin{align*}
\aseq{}{\atomic{t} \\
    &\ \vee \exists g\; \exists h\; t = \cat{g}{h} \\
    &\ \vee \exists g\; \exists k\; t = \enc{g}{k}} \\
\aseq{\atomic{\cat{g}{h}}}{ } \\
\aseq{\atomic{\enc{g}{h}}}{ }
\end{align*}

\item[Predecessors]

\begin{align*}
\aseq{\pred{n}{m}, \pred{m}{p}}{\pred{n}{p}} \\
\aseq{\pred{n}{m}, \pred{m}{n}}{n = m} \\
\aseq{}{\pred{n}{m} \vee \pred{m}{n}} \\
\\
\aseq{\strPred{n}{m}, \strPred{m}{p}}{\strPred{n}{p}} \\
\aseq{\strPred{n}{m}, \strPred{m}{n}}{n = m} \\
\\
\aseq{\strPred{n}{m}}{\pred{n}{m}}
\end{align*}

\item[Direction]

\begin{align*}
\aseq{}{\inbound{n} \vee \outbound{n}} \\
\aseq{\inbound{n}, \outbound{n}}{}
\end{align*}

\item[Keys]

\begin{align*}
\aseq{}{\invinv{k} = k} \\
\aseq{\inv{k} = k}{} \\
\aseq{\sk{A} = \sk{B}}{A = B} \\
\aseq{\pk{A} = \pk{B}}{A = B} \\
\aseq{\inv{\pk{A}} = \inv{\pk{B}}}{A = B} \\
\aseq{\ltk{A, B} = \ltk{C, D}}{A = C, B = D} \\
\aseq{\sk{A} = pk{B}}{} \\
\aseq{\sk{A} = \inv{pk{B}}}{} \\
\aseq{\pk{A} = \inv{\pk{B}}}{}
\end{align*}

%%%%%%%%%%%%%%%%%%%%%%
\subsection{Defined Relations}
%%%%%%%%%%%%%%%%%%%%%%

\item[Outbound Predecessor]
$$ \outPred{n}{m} \equiv \outbound{n}, \pred{n}{m} $$

$$ \outbound{n}, \pred{n}{m} ... $$

\begin{align*}
\aseq{\outbound{n}, \pred{n}{m}}{\outPred{n}{m}}
\end{align*}

\item[Visibly Within]

One term is visibly within another if it is accessible via
deconcatenation alone.

\begin{eqnarray*}
\visSubterm{s}{t} &\equiv& s = t \\
&\indOr& \exists g\; \exists h\; t = \cat{g}{h}, \visSubterm{s}{g} \\
&\indOr& \exists g\; \exists h\; t = \cat{g}{h}, \visSubterm{s}{h}
\end{eqnarray*}

\item[Within]

One term is within another if it is accessible by deconcatenation and
decryption.  So the plaintext of an encryption is within it, but the
encryption key is not.

\begin{eqnarray*}
\subterm{s}{t} &\equiv& \exists g\; \visSubterm{g}{t}, \subterm{s}{g} \\
&\indOr& \exists g\; \exists k\; t = \enc{g}{k}, \subterm{s}{g}
\end{eqnarray*}

\item[Occurrence]

$$ \occ{t}{n} \equiv \subterm{t}{\msg{n}} $$

\item[Availability]

Unavailability is inductive, but availability is not.

\begin{eqnarray*}
\unavail{t} &\equiv& \non{t} \\
&\indOr& \exists n\; \uniq{t}{n} \\
\\
\avail{t} &\equiv& \not \unavail{t}
\end{eqnarray*}

\item[Sent Before]

[Is there a better name?]

$$ \before{t}{n} \equiv
   \exists n'\; \outbound{n'}, \pred{n'}{n}, \occ{t}{n'} $$

\item[Explained In]

A term is explained in a node if a penetrator could construct it from
the messages sent by outbound predecessors of the node.  'explained
in' appears to be inductive only over the 'available' relation.

\begin{eqnarray*}
\expl{t}{n} & \equiv & \avail{t} \\
& \indOr & \exists g\; \exists h\; t = gh, \expl{g}{n}, \expl{h}{n} \\
& \indOr & \exists g\; \exists s\; g = ts, \expl{g}{n} \\
& \indOr & \exists g\; \exists s\; g = st, \expl{g}{n} \\
& \indOr & \exists g\; \exists k\; t = \enc{g}{k}, \expl{g}{n}, \expl{k}{n} \\
& \indOr & \exists g\; \exists k\; g = \enc{t}{k}, \expl{g}{n}, \expl{\inv{k}}{n} \\
& \indOr & \exists n' \outbound{n'}, \inbound{n}, n' \leq n, \occ{t}{n'}
\end{eqnarray*}

\item[Compromised]

What if instead of adding the \emph{unavailable} relation as the
complement of \emph{available}, we add a \emph{compromised} relation
as a partial complement? A term is \emph{compromised} if it is not
only available to an adversary, but actually used by an adversary to
construct an inbound message.

By adding \emph{compromised}, we no longer deal with shapes, but
instead with shapes augmented with a set of compromised
messages. There isn't always a minimum augmented shape.

\begin{align*}
\aseq{}{\expl{t}{n}} \\
\\
\aseq{\expl{t}{n}}{\before{t}{n} \\
    &\ \vee \comp{t} \\
    &\ \vee \exists g\; \exists h\; t = gh, \expl{g}{n}, \expl{h}{n} \\
    &\ \vee \exists g\; \exists s\; g = st, \expl{g}{n} \\
    &\ \vee \exists g\; \exists s\; g = ts, \expl{g}{n} \\
    &\ \vee \exists g\; \exists k\; t = \enc{g}{k}, \expl{g}{n}, \expl{k}{n} \\
    &\ \vee \exists g\; \exists k\; g = \enc{t}{k}, \expl{g}{n},
\expl{\inv{k}}{n}} \\
\\
\aseq{\comp{t}, \unavail{t}}{ }
\end{align*}

\end{description}


%%%%%%%%%%%%%%%%%%%%%%%
\section{Annotations}
%%%%%%%%%%%%%%%%%%%%%%%

\begin{description}

\item[Soundness]
A protocol is \emph{sound} if ``in every execution, whenever a message
is received and its formula is relied upon, there were corresponding
[previous] message transmissions with guaranteed formulas that allow
it to be deduced.''

Ideally, only guarenteed formulas would be written down, and
reliances would be inferred. In this case, a protocol is \emph{sound}
if, in every execution, whenever a message is sent with a guarenteed
formula, there were previous message transmissions with guarenteed
formulas from which it could be deduced.

\end{description}

\end{document}
